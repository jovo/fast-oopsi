\section{Discussion} \label{sec:dis}

% \paragraph{Summary}

Above, we have developed an algorithm to approximate the \emph{maximum a posteriori} (MAP) spike train, given a fluorescence trace.  The approximation is required because finding the actual MAP estimate is not currently computationally tractable.  By replacing the assumed Poisson distribution on spikes with an exponential distribution, we obtain a log-concave optimization problem, which can be solved using standard gradient ascent techniques (such as Newton-Raphson).  This exponential distribution has an advantage over a Gaussian distribution by restricting spikes to be positive, which improves inference quality (c.f. Figure \ref{fig:woopsi_inf}).  This non-negative constraint is enforced by interior-point methods.  Furthermore, by utilizing the special structure of the Hessian matrix (ie, it is tridiagonal), we can very efficiently find this approximate MAP spike train.  Importantly, this algorithm runs fast enough on standard computers that it can be run online.  Finally, all the parameters can be estimated from only the fluorescence observations, obviating the need for joint electrophysiology and imaging (c.f. Figure \ref{fig:woopsi_learn}).  This approach is robust, in that it works ``out-of-the-box'' on all the in vivo and in vitro data that we have (c.f. Figure \ref{fig:woopsi_data4}).

Because this filter is model based, we can generalize the model in several ways to make it more accurate.  Unfortunately, we found that some of these generalizations do not improve inference accuracy, probably because of the exponential approximation.  Instead, the \foopsi filter output can be used to initialize our previously developed SMC filter \cite{VogelsteinPaninski09}, to further improve inference quality (c.f. Figure \ref{fig:smc_init}).  Another model generalization allows us to incorporate spatial filtering of the raw movie into our approach (c.f. Figure \ref{fig:spatial}).  The parameters of the spatial filter can be estimated from the data, even when spatial filters are overlapping (c.f. Figure \ref{fig:spatial_multi_learn}).

A number of extensions follow from this work.  First, further development on some of the model generalizations may improve inference results. Second, putting this filter with a crude but automatic segmentation tool to obtain ROIs would create a completely automatic algorithm that converts raw movies of populations of neurons into populations of spike trains.  Third, combining this algorithm with recently developed connectivity inference algorithms on this kind of data \cite{MishchenkoPaninski09}, could yield very efficient connectivity inference.