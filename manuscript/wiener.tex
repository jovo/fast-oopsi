The Poisson distribution can be replaced with a Gaussian instead of a Poisson distribution, ie,  $n_t \overset{iid}{\sim} \mN(\lam \Del, \lam \Del)$, which, when plugged into Eq. \eqref{eq:nhat2} yields:
\begin{align} \label{eq:obj3}
\hbn &= \argmax_{n_t}  \sum_{t=1}^T \bigg( \frac{1}{2 \sig^2}(F_t - \alpha(C_t + \beta))^2  + 
 \frac{1}{2 \lam \Del}(n_t - \lam \Del)^2\bigg).
\end{align}
Note that since fluorescence integrates over $\Delta$, it makes sense that the mean scales with $\Delta$.  Further, since the Gaussian here is approximating a Poisson with high rate \cite{SjulsonMiesenbock07}, the variance should scale with the mean.  Using the same tridiagonal trick as above, Eq. \eqref{eq:obj3} can be solved using Newton-Raphson once (because its quadratic).  Writing the above in matrix notation, substituting $C_t - \gam C_{t-1}$ for $n_t$, yields:
\begin{align}   \label{eq:w2}
\hbC&= \argmax_{\bC} \frac{1}{2\sig^2} -\norm{\bF - \bC}^2 - \frac{1}{2\lam\Del} \norm{\bM \bC - \lam\Del\ve{1}}^2,
\end{align}
\noindent which is quadratic in $\bC$.  The gradient and Hessian are given by:
\begin{align}
\bg &= -\frac{1}{\sig^2} (\bC - \bF) - \frac{1}{\lam\Del} ( (\bM \hbC)\T \bM + \lam\Del \bM\T \ve{1}), \\
\bH &= \frac{1}{\sig^2} \ve{I} + \frac{1}{\lam\Del} \bM\T \bM.
\end{align}
Note that this solution is the optimal linear solution, under the assumption that spikes follow a Gaussian distribution, and if often referred to as the Wiener filter, regression with a smoothing prior, or ridge regression.  Estimating the parameters for this model follows similarly as above. 