\newpage
\section{Pseudocode} \label{sec:pseudo}

\begin{algorithm}[h!]
\caption{Pseudocode for inferring the approximately most likely spike train, given fluorescence data. Note that the algorithm is robust to small variations $\xi_z, \xi_n$. The equations listed below refer to the most general equations in the text (simpler equations could be substituted when appropriate).  Curly brackets, $\{ \cdot \}$, indicate comments.}
\label{eqn:pseudocode}
\begin{algorithmic}[1]
\STATE initialize parameters, $\bth$ (Section \ref{sec:init})
	
\WHILE{convergence criteria not met}
  \FOR[interior point method to find $\mhb{C}$]{$z=1,0.1,0.01,\ldots, \xi_z$}
    \STATE Initialize $n_t=\xi_n$ for all $t=1,\ldots, T$, $C_1=0$ and $C_t = \gam C_{t-1} + n_t$ for all $t=2,\ldots, T$.
	\STATE let $\mb{C}_{z}$ be the initialized calcium, and $\mh{P}_{z}$, be the posterior given this initialization
	\WHILE[Newton-Raphson with backtracking line searches]{$\mh{P}_{z'}< \mh{P}_{z}$}
		\STATE compute $\mb{g}$ using Eq.~\eqref{eq:g3}
		\STATE compute $\mb{H}$ using Eq.~\eqref{eq:H3}
		\STATE compute $\mb{d}$ using $\mb{H} \backslash \mb{g}$ \COMMENT{block-tridiagonal Gaussian elimination}
		\STATE let $\mb{C}_{z'}=\mb{C}_z + s \mb{d}$, where $s$ is between $0$ and $1$, and $\mh{P}_{z'}>\mh{P}_z$ \COMMENT{backtracking line search}
	\ENDWHILE
  \ENDFOR
\STATE check convergence criteria
\STATE update $\mvb{\alpha}$ and $\mv{\beta}$ using Eq.~\eqref{eq:alpha_x3}  \COMMENT{only if spatial filtering} 
% \STATE update  using Eq.~\eqref{eq:beta}
\STATE let $\sig$ be the root-mean square of the residual
\STATE let $\lam=T/(\Del\sum_t \mh{n}_t)$
\ENDWHILE
\end{algorithmic}
\end{algorithm}

\clearpage
\section{Wiener Filter} \label{sec:wiener}

The Poisson distribution in Eq.~\eqref{eq:n} can be replaced with a Gaussian instead of a Poisson distribution, ie,  $n_t \overset{iid}{\sim} \mc{N}(\lam \Del, \lam \Del)$, which, when plugged into Eq.~\eqref{eq:nhat2} yields:
\begin{align} \label{eq:obj5}
\mhb{n} &= \argmax_{n_t}  \sum_{t=1}^T \bigg( \frac{1}{2 \sig^2}(F_t - \alpha C_t - \beta)^2  + 
 \frac{1}{2 \lam \Del}(n_t - \lam \Del)^2\bigg).
\end{align}
Note that since fluorescence integrates over $\Delta$, it makes sense that the mean scales with $\Delta$.  Further, since the Gaussian here is approximating a Poisson with high rate \cite{SjulsonMiesenbock07}, the variance should scale with the mean.  Using the same tridiagonal trick as above, Eq.~\eqref{eq:obj3} can be solved using Newton-Raphson once (because this expression is quadratic in $\mb{n}$).  Writing the above in matrix notation, substituting $C_t - \gam C_{t-1}$ for $n_t$, and letting $\alpha=1$ yields:
\begin{align}   \label{eq:w2}
\mhb{C}&= \argmax_{\mb{C}} -\frac{1}{2\sig^2} \norm{\mb{F} - \mb{C} - \beta\ve{1}_T}^2 - \frac{1}{2\lam\Del} \norm{\mb{M} \mb{C} - \lam\Del\ve{1}}^2,
\end{align}
\noindent which is quadratic in $\mb{C}$.  The gradient and Hessian are given by:
\begin{align}
\mb{g} &= -\frac{1}{\sig^2} (\mb{C} - \mb{F} - \beta\ve{1}_T) - \frac{1}{\lam\Del} ( (\mb{M} \mhb{C})\T \mb{M} + \lam\Del \mb{M}\T \ve{1}), \\
\mb{H} &= \frac{1}{\sig^2} \ve{I} + \frac{1}{\lam\Del} \mb{M}\T \mb{M}.
\end{align}
Note that this solution is the optimal linear solution, under the assumption that spikes follow a Gaussian distribution, and is often referred to as the Wiener filter, regression with a smoothing prior, or ridge regression \cite{CONV04}.  Estimating the parameters for this model follows similarly as described in Section \ref{sec:learn}.
