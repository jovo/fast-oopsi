\section{Wiener Filter} \label{sec:wiener}

\paragraph{Fast Wiener Filter}

Instead of replacing the Poisson distribution on spikes with an exponential, we can replace it with a Gaussian:

 \begin{align} \label{eq:C3}
	C = \gam C_{t-1} + n_t, \qquad n_t \overset{iid}{\sim} \mN(\lam \Del, \lam \Del)
\end{align}

\noindent which, when plugged into Eq. \eqref{eq:nhat3} yields

\begin{align} \label{eq:obj3}
\bn^{Wiener} &= \argmax_{n_t}  \sum_{t=1}^T \bigg( \frac{1}{2 \sig^2}(F_t - \alpha(C_t + \beta))^2  + 
 \frac{1}{2 \lam \Del}(n_t - \lam \Del)^2\bigg) 
\end{align}

Using the same tridiagonal trick as above, we can solve Eq. \eqref{eq:obj3} using Newton-Raphson once (since we have a quadratic problem here, see Appendix \ref{sec:wiener} for details).  Because we know only positive spikes are possible, at times we will also consider, $[$Wiener$]_+$, which is the Wiener filter half-wave rectified, i.e., all sub-zero values are set to zero.

Sections \ref{sec:inf} outline one approach to solving Eq. \eqref{eq:obj}, by approximating the Poisson distribution with an exponential distribution, and imposing a non-negative constraint on the inferred $\hbn$.  Perhaps a more straightforward approach would be to approximate the Poission distribution with a Gaussian distribution.  In fact, as rate increases above about $10$ spikes/sec, a Poisson distribution with rate $\lam \Del$ is well approximated by a Gaussian with mean and variance $\lam \Del$.  Given such an approximation, instead of Eq. \eqref{eq:obj2}, we would obtain:

\begin{align} \label{eq:w1}
\hbn_{w} &\approx \argmin_{n_t \in \Real, \forall t} \sum_{t=1}^T \bigg(\frac{1}{2\sig^2} (F_t-C_t)^2 + \frac{1}{2 \lam \Del} (n_t - \lam \Del)^2\bigg)% \nonumber \\
\end{align}

\noindent As above, we can rewrite Eq. \eqref{eq:w1} in matrix notation in terms of $\bC$:

\begin{align}   \label{eq:w2}
\hbC_{w}&= \argmin_{C_t \in \Real, \forall t} \frac{1}{2\sig^2} \norm{\bF - \bC}^2 + \frac{1}{2\lam\Del} \norm{\bM \bC - \lam\Del\ve{1}}^2 
\end{align}

\noindent which is quadratic in $\bC$, and may therefore be solved analytically using quadratic programming, $\hbC_w = \hbC_0 + \bd_w$, where $\hbC_0$ is the initial guess and $\bd_w=\bH_w \backslash \bg_w$, where

\begin{align}
\bg_w &= \frac{1}{\sig^2} (\bC_0' - \bF) + \frac{1}{\lam\Del} ( (\bM \hbC_0)' \bM - \lam\Del \bM' \ve{1}) \\
\bH_w &= \frac{1}{\sig^2} \ve{I} + \frac{1}{\lam\Del} \bM' \bM
\end{align}

Note that this solution is the optimal linear solution, under the assumption that spikes follow a Gaussian distribution, and if often referred to as the Wiener filter, regression with a smoothing prior, or ridge regression.  To estimate the parameters for the Wiener filter, we take the same approach as above:

\begin{subequations}
\begin{align} \label{eq:w_par}
\hbth_w &\approx \argmax_{\bth_w}P[\ve{F}| \hbn_w, \thet_w) P[\hbn_w | \thet_w) \\
%&= \argmax_{\bth_w} \sum_{t=1}^T \left(-\frac{1}{2} \log (2\pi\sig^2)-\frac{1}{2\sig^2} (F_t - \gamma \hC_{t-1} - \hn_t - \beta)^2 \right) \nonumber \\
%&\qquad \qquad \qquad + \sum_{t=1}^T \bigg(-\frac{1}{2} \log (2\pi\lam\Del)-\frac{1}{2\lam\Del} (\hn_t - \lam\Del)^2 \bigg) \\
&= \argmax_{\bth_w} -\frac{T}{2} \log (4\pi^2\sig^2\lam\Del) - \frac{1}{2\sig^2} \norm{\bY_w + \ve{\eta}_w \bX_w}^2 - \frac{1}{2\lam\Del} \norm{\hbn_w - \lam\Del\ve{1}}^2
\end{align}
\end{subequations}

\noindent where $\bY_w$, $\ve{\eta}_w$, and $\bX_w$ are defined as their subscriptless counterparts in Eq. \eqref{eq:par1}.


