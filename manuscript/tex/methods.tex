As described above, to develop an algorithm to approximate the most likely spike train given fluorescence data, we first carefully analyze the statistics of typical data-sets.  We start by considering an in vitro experiment, for which the SNR is relatively high, and build an appropriate generative model (Section \ref{sec:model}).  Given this model, we can formally state our goal (Section \ref{sec:goal}).  And given this goal, we derive an approximately optimal inference algorithm (Section \ref{sec:inf}).  We then generalize our model in a number of ways, incorporating spatial filters (Section \ref{sec:spatial}), overlapping spatial filters (Section \ref{sec:overlap}), Poisson observations (Section \ref{sec:poisson}), fluorescence saturation (Section \ref{sec:satur}), and slow rise time for genetic sensors (Section \ref{sec:slow}).  Inferring the most likely spike train in all the above scenarios requires having an estimate of the parameters governing the relationship between the spikes and the movie.  Thus, we also develop an approach to efficiently approximate the maximum likelihood estimate (MLE) of the parameters (Section \ref{sec:learn}).  Finally, we describe several measures we use to assess and compare performance of our algorithm with others (Section \ref{sec:ass}).    