\paragraph{Slice Preparation and Imaging}

All animal handling and experimentation was done according to the National Institutes of Health and local Institutional Animal Care and Use Committee guidelines. Somatosensory thalamocortical slices $400$ $\mu$m thick were prepared from C57BL/6 mice at age P14 as described \cite{MacLeanYuste05}. Neurons were filled with $50$ $\mu$M Fura $2$ pentapotassium salt (Invitrogen, Carlsbad, CA) through the recording pipette. Pipette solution contained $130$ K-methylsulfate, $2$ MgCl$_2$, $0.6$ EGTA, $10$ HEPES, $4$ ATP-Mg, and $0.3$ GTP-Tris, pH $7.2$ ($295$ mOsm).  After cells were fully loaded with dye, imaging was done by using a modified BX50-WI upright confocal microscope (Olympus, Melville, NY).  Image acquisition was performed with the C9100-12 CCD camera from Hamamatsu Photonics (Shizuoka, Japan) with arclamp illumination at $385$ nm and $510/60$ nm collection filters (Chroma, Rockingham, VT).  Images were saved and analyzed using custom software written in Matlab (Mathworks, Natick, MA).

\paragraph{Electrophysiology}

All recordings were made using the Multiclamp 700B amplifier (Molecular Devices, Sunnyvale, CA), digitized with National Instruments 6259 multichannel cards and recorded using custom software written using the LabView platform (National Instruments, Austin, TX) .  Waveforms were generated using Matlab and were given as current commands to the amplifier using the LabView and National Instruments system. The shape of the waveforms mimicked excitatory (inhibitory) synaptic inputs, with a maximal amplitude of $+70$ pA ($-70$ pA).
