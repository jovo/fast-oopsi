%Experiments often yield measurements of variables that are naturally constrained to be nonnegative. In such scenarios, it may be desirable to filter (or deconvolve) the observations to find the most likely trajectory of the nonnegative variable, given the noisy observations. Here, we develop a computationally-efficient optimal filter for a certain subset of nonnegatively constrained deconvolutions.  Specifically, for any nonnegative variable that is filtered by a matrix linear differential equation and observed with independent, log-concave noise, we can infer the optimal nonnegative trajectory via straightforward interior-point methods in $O(T)$ (linear) time, as opposed to more standard approaches requiring $O(T^3)$ time (where $T$ is the total number of time steps).  The key is to make use of the tridiagonal structure of the Hessian of the log-posterior here, which allows us to perform each Newton iteration in linear time.  We apply this filter to an important problem in neuroscience: inferring a spike train from noisy calcium fluorescence observations. We demonstrate the filter's improved performance on simulated and real data. In conclusion, we propose that this filter is readily applicable for a number of real-time applications, including spike inference from simultaneously-observed large neural populations.

A fundamental desideratum in neuroscience is to simultaneously observe the spike trains from large populations of neurons. Calcium imaging technologies are bringing the field ever closer to achieving this goal, both in vitro and in vivo. To get the most information out of these preparations, one can increase the frame rate and image field, leading to corresponding increases in temporal resolution and number of observable cells.  However, these increases come at the cost of reducing the dwell time per pixel, causing a decrease in the signal-to-noise ratio.  Thus, to maximize the utility of these technologies, powerful computational tools must be built to compliment the experimental tools.  In particular, by considering the statistics of the data --- e.g., firing rates and photon counts are positive (or zero) --- we develop an approximately optimal algorithm for inferring spike trains from fluorescence data. More specifically, we use an interior-point method to perform a non-negative deconvolution, inferring the approximately most likely spike train for each neuron, given their fluorescence signals. We demonstrate using simulations, in vitro, and in vivo data-sets the improvement of our algorithm over other techniques.  Moreover, because our inference is very fast --- requiring only about 1 second of computational time on laptop to analyze a calcium trace from 50,000 image frames --- we call this approach the Fast Non-negative deconvolution Spike Inference (\foopsi) filter. We demonstrate that performing optimal spatial filtering on the images further refines the estimates.  Importantly, all the parameters required to perform our inference can be estimated using only the fluorescence data, obviating the need to perform simultaneous electrophysiological experiments.  Finally, all the code written to perform the inference is freely available.


