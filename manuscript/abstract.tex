\begin{abstract}
Calcium imaging technologies for observing spiking activity simultaneously from large populations of neurons are quickly gaining popularity.  While the raw data are fluorescence movies, often, the underlying spike trains are of interest.  We develop here an online non-negative deconvolution filter to infer the approximately mostly likely spike trains for each neuron, given the fluorescence observations.  This algorithm outperforms optimal linear deconvolution (aka, Wiener filter) on both simulated and in vitro data, and requires approximately the same computational time.  The performance gains come from restricting the inferred spike trains to be positive (using an interior-point method), unlike the Wiener filter.  The algorithm is fast enough that even when simultaneously imaging $\approx 100$ neurons, inference can simultaneously be performed on all observed traces faster than real time.  We demonstrate that performing optimal spatial filtering on the images further refines the estimates.  Importantly, all the parameters required to perform our inference can be estimated using only the fluorescence data, obviating the need to perform simultaneous electrophysiological calibration experiments.
%A fundamental desideratum in neuroscience is to simultaneously observe the spike trains from large populations of neurons. Calcium imaging technologies are bringing the field ever closer to achieving this goal, both in vitro and in vivo. To get the most information out of these preparations, one can increase the frame rate and image field, leading to corresponding increases in temporal resolution and number of observable cells.  However, these increases come at the cost of reducing the dwell time per pixel, causing a decrease in the signal-to-noise ratio.  Thus, to maximize the utility of these technologies, powerful computational tools must be built to compliment the experimental tools.  In particular, by considering the statistics of the data --- e.g., spikes are positive --- we develop an approximately optimal algorithm for inferring spike trains from fluorescence data. More specifically, we use an interior-point method to perform a fast non-negative deconvolution, inferring the approximately most likely spike train for each neuron, given their fluorescence signals. We demonstrate using simulations and in vitro data-sets the improvement of our algorithm over optimal linear deconvolution.  Moreover, because our inference is very fast --- requiring only about 1 second of computational time on laptop to analyze a calcium trace from 50,000 image frames --- we call this approach the \foopsi filter. We demonstrate that performing optimal spatial filtering on the images further refines the estimates.  Importantly, all the parameters required to perform our inference can be estimated using only the fluorescence data, obviating the need to perform simultaneous electrophysiological experiments.  Finally, all the code written to perform the inference is freely available from the authors upon request.
\end{abstract}