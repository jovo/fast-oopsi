\section{Wiener Deconvolution} \label{sec:wiener}

We can write the above model as a convolution:

\begin{align}
F(t) = y(t) \ast n(t) + \varepsilon(t)
\end{align}

\noindent where $y(t)=Ae^{-t/\tau}$ is the calcium convolution kernel. The goal is then to find some optimal deconvolution kernel, $g(t)$, so that we may estimate $n_t$ as follows:

\begin{align} \label{eq:opt_n}
\widehat{n}(t) = g(t) \ast F(t).
\end{align}

The Wiener deconvolution filter is most easily described in the frequency domain:

\begin{align} \label{eq:G}
G(f) = \frac{|Y^(f)| E[N(f)]^2}{|Y(f)|^2 E[N(f)]^2 + E[\epsilon(f)]^2}
\end{align}

\noindent where we use the notation $X(f)$ to indicate the Fourier transform of $x(t)$, and $E[X(f)]^2$ to indicate the power spectral density (PSD) of $x(t)$.  By assuming that $n(t)$ is Poisson (or binomial), and $\varepsilon(t)$ is Gaussian, we can write their PSD's analytically:

\begin{align} \label{eq:psdN}
E[N(f)]^2 &=\\ \label{eq:psde}
E[\epsilon(f)]^2&=.
\end{align}

Similarly, because $y(t)$ is an exponential, we may write its Fourier transform as:

\begin{align} \label{eq:Y}
Y(f)=.
\end{align}

Thus, plugging in \eqref{eq:psdN},\eqref{eq:psde}, and \eqref{eq:Y} into \eqref{eq:G}, we have

\begin{align}
G(f)&=.
\end{align}

Finally, taking the inverse Fourier transform of $G(f)$, we are left with

\begin{align}
g(t)=,
\end{align}

\noindent which we may plug back into \eqref{eq:opt_n} to find $\widehat{n}(t)$.


\subsection{Lin Alg way}

Weiner deconv of exponential kernel.

First note that the $\Ca_t$ is simply a convolution of the spike train with a linear kernel,

\begin{align}
\ve{\Ca} = Ae^{-\ve{t}/\tau_c} \ast \ve{n}
\end{align}

\noindent where $\ve{t}$ indicates the vector, $[A, Ae^{-\Delta/\tau_c},\ldots, Ae^{-L\Delta/\tau_c}]$, with $L$ the length of this linear kernel.  The spike train may also be written as a linear function of $\Ca_t$:


\begin{align}
\ve{n} = \ma{Y} \ve{\psi} + \ve{\varepsilon}
\end{align}

\noindent where $\ma{Y}$ is a $T \times L$ matrix

\begin{equation}
\begin{bmatrix}
\Ca_0&\Ca_1&\ldots&\Ca_{L-1}&1\\
\Ca_1&\Ca_2&\ldots&\Ca_L&1\\
\vdots\\
\Ca_{T-L}&\Ca_{T-L+1}&\ldots&\Ca_T&1
\end{bmatrix}.
\end{equation}

The solution for the optimal $\ve{\psi}$ is thus given by,

\begin{align}
E[\ve{\psi}] &= E[(\ma{Y}^T\ma{Y})^{-1}(\ma{Y}^T\ve{n})]\\
&=E[\ma{Y}^T\ma{Y}]^{-1}E[\ma{Y}^T\ve{n}]
\end{align}

Importantly, $\ma{Y}$ may be written as a linear function of $\ve{n}$:

\begin{align} \label{eq:YKn}
\ma{Y}&=\ma{K}\ve{n}+\ve{\varepsilon}\\
\Rightarrow E[\ma{Y}] &= \ma{K}E[\ve{n}]
\end{align}

\noindent where $\ma{K}$ is the convolution kernel matrix:

\begin{equation}
\begin{bmatrix}
A&0&\ldots&0\\
Ae^{-\Delta/\tau_c}&A&0&\ldots\\
Ae^{-2\Delta/\tau_c}&Ae^{-\Delta/\tau_c}&A&\ldots\\
\vdots\\
Ae^{-T\Delta/\tau_c}&\ldots&\ldots&A
\end{bmatrix}.
\end{equation}

Thus, \eqref{eq:YKn} may be expanded

\begin{align}
E[\ma{Y}^T\ma{Y}]^{-1}E[\ma{Y}^T\ve{n}] &= (\ma{K}E[\ve{n}])^T(\ma{K}E[\ve{n})]^{-1} (\ma{K}E[\ve{n}])^T E[\ve{n}]\\
&=\big((E[\ve{n}]^T\ma{K}^T)(\ma{K}E[\ve{n}])\big)^{-1}E[\ve{n}^T]\ma{K}^TE[\ve{n}].
\end{align}

Converting to the Fourier domain, we have

\begin{align}
\ma{K} &= \frac{1}{1+a\omega}\\
\ma{K}^T\ma{K} &= \frac{1}{1+a\omega^2}\\
\ma{K}^T\ma{K}+b\ma{I} &= b+\frac{1}{1+a\omega^2}\\
E[\ve{n}] &= \lambda
\end{align}

