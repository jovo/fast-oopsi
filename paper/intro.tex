\paragraph{Motivation}

Simultaneously imaging large populations of neurons using calcium sensors is becoming increasingly popular, both in vitro  \cite{YusteKonnerth06} and in vivo \cite{NagayamaChen07, GobelHelmchen07, LuoSvoboda08}, especially as the signal-to-noise-ratio (SNR) of genetic sensors continues to improve \cite{GaraschukKonnerth07, MankGriesbeck08, WallaceHasan08}. Whereas, the data from these experiments are movies of time-varying fluorescence signals, the desired signal is typically the spike trains or firing rates of the observable neurons.  Importantly, to a first approximation, somatic calcium concentration has a relatively simple relationship to spikes.  Thus, in theory, one could infer the most likely spike train of each neuron, given the fluorescence data.  

\paragraph{Limitations of calcium imaging}

Unfortunately, finding the \emph{most} likely spike train is a challenging computational task, for a number of reasons.  First, the signal-to-noise ratio (SNR) is often low, especially as one increases the image field and frame rate.  Second, to find the most likely spike train for a given fluorescence signal, one would have to search over all possible spike trains, a search that would take far too long, pragmatically.  

\paragraph{Computational tools as important as experimental tools}

One is therefore effectively forced to find an approximately most likely spike train, or guess that the inferred spike train is most likely (but not really be sure).  The precise details of these approximations, however, are crucial, especially as one approaches shot-noise limited data. For in vitro studies, one often uses 1-photon imaging --- either confocal or widefield --- in which case the number of photons per neuron is a function of magnification and frame rate (obviously, other parameters, such as number of sensors per neuron, are also important, but typically more difficult to control).  To maximize the amount of information one can extract from such preparations, one should increase the frame rate and image field until achieving the shot noise limited regime, assuming one has an inference algorithm that can operate in such a scenario.  For in vivo studies, one often uses 2-photon laser scanning microscopy, for which the SNR is relatively reduced,  because the dwell time per pixel is (frame duration)/(number of pixels in frame).  To circumvent the low SNR for in vivo studies, some groups use long integration times (e.g., \cite{OhkiReid06}), whereas others use small image fields (e.g., \cite{KerrHelmchen07}).  One would prefer to neither sacrifice temporal resolution nor number of observable neurons, to get sufficient signal quality to perform reliable inference.  Therefore, it is of the utmost importance to extract as much information as possible from the signal; especially if one is asking quantitative questions about the statistics of the spike trains from the observable neurons --- either spontaneously or in relation to some sensorimotor stimulus.

\paragraph{Previous approaches}

%While this technology facilitates acquiring data from an unprecedented number of neurons simultaneously, it is not yet a panacea: there is a trade-off between the quantity of neurons, and the quality of data. While one can now acquire data from $\sim 100$ neurons within a single experimental session, the data for each neuron is relatively poor. In comparison with extracellular electrophysiology, the data from calcium imaging has reduced (i) SNR and (ii) temporal resolution. 

A number of groups have therefore proposed algorithms to infer spike trains from calcium fluorescence data.  For instance, Greenberg et al. \cite{GreenbergKerr08} developed a novel template matching algorithm, which performed well on their data, but might not be quite as effective as the noise increased, because it relies on a relatively high SNR to not require unreasonable amounts of processing power.  Holekamp et al. \cite{HolekampHoly08} took a very different strategy, by performing the optimal linear deconvolution (i.e., the Wiener filter) on the fluorescence data.  This approach is natural from a signal processing standpoint, but does not carefully consider the statistics of the data.  More specifically, their algorithm allows for both negative spikes and negative photon counts, neither of which are possible.  In our previous work \cite{VogelsteinPaninski09b}, we developed a sequential Monte Carlo method to efficiently compute the approximate probability of a spike in each image frame, given the entire fluorescence time series.   While effective, that approach is not suitable for online analyses, as the computations run in approximately real-time (i.e., analyzing one minute of data requires about one minute of computational time).

\paragraph{Our approach}

The present work takes the following approach.  First, we carefully consider the statistics of typical data sets, and then write down a generative model that accurately relates spiking to observations. Unfortunately, inferring the most likely spike train given this model is computationally intractable.  We therefore make some well-justified approximations, which lead to an algorithm that infers the approximately most likely spike train, given the fluorescence data.  Our algorithm has a few particularly noteworthy features, relative to other approaches.  First, we assume that spikes are always non-negative (i.e., either positive or zero).  This is often an important assumption when searching for non-negative signals \cite{LeeSeung99, LeeSeung01, HuysPaninski06}.  Second, our algorithm is extremely fast: it can process a calcium trace from 50,000 images in about one second on a standard laptop computer.  Because of these two features, we call our approach the FAst Non-negative deconvolution Spike Inference (FANSI) filter. In addition to these two features, we can generalize our model in a number of ways, to incorporate spatial filtering of the images, overlapping neurons, poisson observations (for use with in vivo data), and slow rise time (for use with genetic sensors).  


%Thus, relatively few studies have been able to use this approach to date to ask \emph{quantitative} questions about the relationship between spike trains, and, for example, sensory stimuli \cite{StosiekKonnerth03, OhkiReid05, OhkiReid06, YaksiFriedrich07, SatoSvoboda07, KerrHelmchen07, OzdenWang08}.  

%The present work also develops a deconvolution filter, but has several advantages over the Wiener filter.  First, we impose a non-negative constraint on the solution. Because we know that spikes are non-negative events, this is a desirable constraint to impose on our inference procedure.  In practice, this constraint regularizes the resulting inference \cite{LeeSeung99, LeeSeung01, HuysPaninski06}, by combating against ``ringing'' overfitting effects. Second, the computational time of our filter scales linearly with $T$, whereas the Wiener filter typically scales according to $T \log T$.  Thus, we have named our approach the FAst Non-negative deconvolution Spike Inference (FANSI) filter.  This FANSI filter could be applicable in a number of scientific investigations from myriad fields, and is closely related to the problem of non-negative matrix factorization, a problem currently receiving much attention from the machine learning community \cite{PortugalVicente94, LeeSeung99, LeeSeung01, LeeNg06, OGradyPearlmutter06}.

%The remainder of this paper is organized as follows.  In Section \ref{sec:methods}, we describe our parametric model, and the FANSI filter.  Section \ref{sec:results} first provides our main result: we can utilize FANSI to filter calcium fluorescence data, with better results than the optimal linear filter.  We then generalize the FANSI filter in a number of directions.  First, instead of assuming the data is a 1-dimensional fluorescence time series, we operate on  all the pixels within a region-of-interest (ROI), which can significantly improve the SNR of our inference. Second, by embedding our FANSI filter into a pseudo-expectation-maximization algorithm, we can learn the parameters of our model without any training data (i.e., without requiring simultaneous imaging and electrophysiology).  To demonstrate the utility of this approach, we compare our filter and the Wiener filter's output on in vitro data. Third, when imaging a large population of neurons simultaneously, sometimes segmenting the image to obtain one neuron per ROI is difficult.  Therefore, we show how approach can be further generalized to deal with such a scenario.  We then apply our filter to a movie of $\sim 50$ neurons recorded simultaneously in vitro, to indicate that this approach works ``out-of-the-box'' on segmented data.  Fourth, we modify our model to handle data collected in vivo using genetic sensors.  Specifically, this means allowing for a slow rise time and poisson observations.  Finally, in the discussion, we show how to incorporate non-linear saturation into the FANSI filter, and discuss how the output of our FANSI filter can be used to initialize our more powerful (but slower) algorithms.