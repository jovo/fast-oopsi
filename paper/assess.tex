Let $\hbn$ be the inferred spike train, using some algorithm, and $\bn$ be the true spike train.  When spiking is sparse (i.e., at most, one spike per frame), a reasonable measure of inference quality is the effective signal-to-noise ratio, which we define as the squared size of $\hbn$ when there is a spike, divided by the squared size of $\hbn$ when there is no spike:

\begin{align}
	% \rho_{\hbn} &= \frac{\sig_{\bn \hbn}}{\sig_{\bn}\sig_{\hbn}} \label{eq:R}\\
	% \text{MSE}_{\hbn} &= \frac{1}{T} \sum_t (n_t - \hn_t)^2 \label{eq:MSE}\\
	\text{eSNR}_{\hbn} &= \frac{\sum_{t | n_t=1} \hn_t}{\sum_{t | n_t=0} \hn_t} \label{eq:SNR} %
%	\\ \text{AUC}_{\hbn} &= \sum_{k \in \mK} \frac{\sum_{t | n_t=1} I\{\hn_t>k\}}{\sum_{t | n_t=1} I\{n_t=1\}} \label{eq:AUC}
\end{align}

When the neuron is emits many spikes per frame, eSNR is not meaningful.  Instead, we compute the mean squared error between the magnitude of inferred spikes and the true spikes:

%We compute the following four measures of inference quality (a) correlation coefficient, (b) mean square error, and (c) signal-to-noise ratio: %, and (d) area under the ROC curve:

%\begin{subequations} \label{eq:stats}
\begin{align}
%	\rho_{\hbn} &= \frac{\sig_{\bn \hbn}}{\sig_{\bn}\sig_{\hbn}} \label{eq:R}\\
	\text{MSE}_{\hbn} &= \frac{1}{T} \sum_t (n_t - \hn_t)^2 \label{eq:MSE}
%	\\ \text{SNR}_{\hbn} &= \frac{\sum_{t | n_t=1} \hn_t}{\sum_{t | n_t=0} \hn_t} \label{eq:SNR} %
%	\\ \text{AUC}_{\hbn} &= \sum_{k \in \mK} \frac{\sum_{t | n_t=1} I\{\hn_t>k\}}{\sum_{t | n_t=1} I\{n_t=1\}} \label{eq:AUC}
\end{align}
%\end{subequations}

%\noindent where $\sig_{\bn \hbn}$ is the covariance between $\bn$ and $\hbn$ and $\sig_{\hbn}$ is the variance of $\hbn$, $I\{x\}=1$ if $x$ is true, and $0$ otherwise, and $\mK=\{0,0.1,0.2,\ldots 1\}$.  Note that SNR and AUC are only appropriate assuming that there is at most one spike within an image frame. 