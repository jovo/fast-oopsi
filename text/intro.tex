Simultaneously imaging large populations of neurons using calcium sensors is becoming increasingly popular \cite{YusteKonnerth06, NagayamaChen07, GobelHelmchen07, LuoSvoboda08}, especially as the signal-to-noise-ratio (SNR) of genetic sensors continues to improve \cite{GaraschukKonnerth07, MankGriesbeck08, WallaceHasan08}. While this technology facilitates acquiring data from an unprecedented number of neurons simultaneously, it is not yet a panacea: there is a trade-off between the quantity of neurons, and the quality of data. While one can now acquire data from $\sim 100$ neurons within a single experimental session, the data for each neuron is relatively poor. In comparison with extracellular electrophysiology, the data from calcium imaging has reduced (i) SNR and (ii) temporal resolution. Thus, relatively few studies have been able to use this approach to date to ask \emph{quantitative} questions about the relationship between spike trains, and, for example, sensory stimuli \cite{StosiekKonnerth03, OhkiReid05, OhkiReid06, YaksiFriedrich07, SatoSvoboda07, KerrHelmchen07, OzdenWang08}.  

A number of groups have proposed spike inference algorithms to facilitate using calcium imaging to ask quantitative questions about spike timing. Finding the most likely spike train, given a calcium movie, is computationally intractable because we have to perform a search over all possible spike trains, and the number of possible spike trains scales exponentially with the number of image frames, $T$ (more specifically, assuming only a single spike can occur per frame, we have $2^T$ possible spike trains).  To circumvent this problem, various groups have developed distinct strategies.  For instance, Greenberg et al. \cite{GreenbergKerr08} reduced the search space by establishing heuristics to preprocess the data, and effectively exclude many image frames from the search space.  In our previous work \cite{VogelsteinPaninski09b}, we developed a sequential Monte Carlo method to efficiently sample spike trains given the fluorescence data, which is guaranteed to perform optimally given our model, which incorporates saturation, refractoriness, and stimulus dependent effects (this approach has the added advantage of providing a measure of uncertainty as well). Unfortunately, both these approaches are relatively slow, as they still require searches over large spaces.  Holekamp et al. \cite{HolekampHoly08} took a very different strategy, by performing the optimal linear deconvolution (i.e., the Wiener filter) on the fluorescence data. This approach may be thought of as finding the \emph{maximum a posteriori} (MAP) spike train.  

The present work also develops a deconvolution filter, but has several advantages over the Wiener filter.  First, we impose a non-negative constraint on the solution. Because we know that spikes are non-negative events, this is a desirable constraint to impose on our inference procedure.  In practice, this constraint regularizes the resulting inference \cite{LeeSeung99, LeeSeung01, HuysPaninski06}, by combating against ``ringing'' overfitting effects. Second, the computational time of our filter scales linearly with $T$, whereas the Wiener filter typically scales according to $T \log T$.  Thus, we have named our approach the FAst Non-negative Deconvolution (FAND) filter.  This FAND filter could be applicable in a number of scientific investigations from myriad fields, and is closely related to the problem of non-negative matrix factorization, a problem currently receiving much attention from the machine learning community \cite{PortugalVicente94, LeeSeung99, LeeSeung01, LeeNg06, OGradyPearlmutter06}.

The remainder of this paper is organized as follows.  In Section \ref{sec:methods}, we describe our parametric model, and the FAND filter.  Section \ref{sec:results} first provides our main result: we can utilize FAND to filter calcium fluorescence data, with better results than the optimal linear filter.  We then generalize the FAND filter in a number of directions.  First, instead of assuming the data is a 1-dimensional fluorescence time series, we operate on  all the pixels within a region-of-interest (ROI), which can significantly improve the SNR of our inference. Second, by embedding our FAND filter into a pseudo-expectation-maximization algorithm, we can learn the parameters of our model without any training data (i.e., without requiring simultaneous imaging and electrophysiology).  To demonstrate the utility of this approach, we compare our filter and the Wiener filter's output on in vitro data. Third, when imaging a large population of neurons simultaneously, sometimes segmenting the image to obtain one neuron per ROI is difficult.  Therefore, we show how approach can be further generalized to deal with such a scenario.  We then apply our filter to a movie of $\sim 50$ neurons recorded simultaneously in vitro, to indicate that this approach works ``out-of-the-box'' on segmented data.  Fourth, we modify our model to handle data collected in vivo using genetic sensors.  Specifically, this means allowing for a slow rise time and poisson observations.  Finally, in the discussion, we show how to incorporate non-linear saturation into the FAND filter, and discuss how the output of our FAND filter can be used to initialize our more powerful (but slower) algorithms.