\documentclass[12pt]{article}
\usepackage{a4wide}
%\usepackage{apalike}
\usepackage{epsfig}
\usepackage{amsmath}
\usepackage{amsthm}
\pagestyle{empty}
\input{../../my-latex-defs-no-ams}
%\topmargin -0.3in		% in addition to 1.5'' standard margin
%\textheight 8.7in		% 11 - ( 1.5 + \topmargin +  <bottom-margin> )
\newcommand{\vs}{\vspace{5mm}}

\topmargin -0.5in  % in addition to 1.5'' standard margin
\textheight 9in  % 11 - ( 1.5 + \topmargin +  <bottom-margin> )
\oddsidemargin 0.0in  % in addition to 1'' standard
\textwidth 6.5in  % 8.5 - 2 * ( 1 + \oddsidemargin )

\tolerance=1000

\newcommand{\projecttitle}{Optimal inference of spike times, calcium dynamics,
and voltage given noisy, intermittently-sampled fluorescence imaging data}

\newcommand{\Cat}{[Ca\ensuremath{^{2+}]_t} }

\begin{document}
\pagestyle{plain}		%most pages have page numbers
\thispagestyle{empty}		%but no page number on summary page
\setcounter{page}{0}

\begin{center}
THE MCKNIGHT ENDOWMENT FUND FOR NEUROSCIENCE \\
APPLICATION FOR THE MCKNIGHT TECHNOLOGICAL INNOVATIONS IN NEUROSCIENCE AWARD
\end{center}

\vspace{1cm}

\textbf{Name}: Liam Paninski

\textbf{Department}: Department of Statistics and Center for Theoretical Neuroscience

\textbf{Position}: Associate Professor (as of July 2008)

\textbf{Address}: Department of Statistics and Center for Theoretical
Neuroscience, 1255 Amsterdam Ave, New York, NY 10027

\textbf{Phone}: 212-851-2166

\textbf{Fax}: 212-851-2164

\textbf{E-mail}: \email

\textbf{Title}: \projecttitle

\textbf{Project period}: August 1, 2008 to July 31, 2010

\textbf{Responsible financial officer}: Patricia Valencia, Research
Administration, Columbia University, 254 Engineering Terrace, Mail
Code 2205, 1210 Amsterdam Avenue, New York, NY 10027; Phone: (212)
854-0371; Fax: (212) 854-2738


\clearpage


\begin{center}
\textbf{\projecttitle}

\textbf{McKnight Technological Innovations in Neuroscience Proposal}

\textbf{Liam Paninski}
\end{center}

\noindent \textbf{Summary}: We are developing optimal statistical
methods to extend the reach of calcium- and voltage-sensitive imaging
technology.

\noindent \textbf{Problem statement}: The march of progress in imaging
techniques in neuroscience over the past decade has been impressive.
Recent advances in calcium sensing technologies enable us to
simultaneously image the activity of many neurons both in vitro and in
vivo \cite{CAR03,Kerr05,OhkiReid06,DombeckTank07}.  Similarly,
voltage-sensitive imaging techniques now provide measurements of
time-varying voltage at sub-cellular spatial and sub-millisecond
temporal resolution
\cite{Dombeck04,NURIYA06,Sacconi06,Canepari07,Fisher08,Djurisic08}.
This opens up an exciting range of opportunities for better
understanding the dynamics of neural populations and of dendritic
computation.  However, a number of obstacles remain before we can take
full advantage of these opportunities.

\begin{enumerate}
  
\item We can not observe either calcium or voltage directly via
  imaging techniques.  Instead, we must contend with noisy
  observations of the signals of interest.  Low signal-to-noise is
  currently a particular problem in voltage-sensing approaches to
  investigating dendritic computation.

\item Observations across a large spatial field (e.g., many neurons or
  many sites on a dendritic tree) are temporally subsampled, due to
  relatively slow scan rates when using 2-photon or confocal
  microscopy.

\item Calcium dynamics are much slower than action potentials; thus,
  to infer the precise spike times of a neuron, we must perform some
  kind of deconvolution on the observed calcium signal, and
  deconvolution typically further increases noise.

\item There is often a nonlinear relationship between the observed
  fluorescence signal and the underlying voltage or calcium
  concentration.

\item The dynamics of voltage and calcium are often nonlinear and
non-Gaussian.

\item Although in many cases we have good models for describing these
  calcium or voltage dynamics (e.g., \cite{TankDelaney95}), most of
  the parameters governing these models are unknown; these parameters
  vary from neuron to neuron, or may vary spatially as a function of
  distance from the soma.

\end{enumerate}

These factors imply that sophisticated methods must be developed to
infer the true spike times or voltage dynamics given noisy,
single-trial imaging data.  Fortunately, similar problems arise in a
wide variety of applications, and a number of effective methods have
been developed in the statistical literature to overcome these
obstacles.  

\clearpage

\noindent \textbf{Specific Aims}: We have three major complementary
goals in this project.  We intend to develop mathematical methods and
efficient, robust, publicly-available software for:

\begin{enumerate}
  
\item Inference of spike timing and calcium dynamics from noisy,
  saturating, subsampled, large-scale multineuronal calcium-sensitive
  optical recordings.

\item Filtering and biophysical parameter estimation given noisy,
  subsampled spatiotemporal voltage-sensitive imaging data.

\item Estimation of spatiotemporal voltage dynamics given noisy,
  subsampled calcium-sensitive imaging data.

\end{enumerate}

We will describe each of these aims in more detail below, after
providing an overview of the proposed optimal filtering methodology
necessary to achieve these three aims.



\begin{figure}[t!]
\begin{center}
\epsfxsize=3.5in
\epsffile{hmm-schematic.eps}
\caption{Schematic overview of the hidden Markov modeling paradigm.}
\label{fig:hmm-schematic}
\end{center}
\end{figure}

\noindent \textbf{Overview of the proposed methodology}: Each of our
three main goals may be conveniently cast in the language of hidden
Markov systems.  The basic insight common to all of the applications
we will discuss here is that the underlying biological system may be
described as a stochastic dynamical system: a multidimensional state
variable $h(t)$ evolves through time according to some probabilistic
dynamics $p[h(t)|h(t-1),\theta]$, where the dynamics are known up to a
few parameters $\theta$.  For example, in the voltage-sensing
applications discussed below, the state variable $h(t)$ will include
the spatiotemporal voltage on the dendritic tree, and the dynamics
$p[h(t)|h(t-1)]$ will encode the right-hand-side of the cable
equation, driven by stochastic noise sources.

Now we do not observe the state variable $h(t)$ directly (in fact, the
letter $h$ was chosen to denote ``hidden''); instead, our observations
$y(t)$ are a noisy, subsampled version of $h(t)$, summarized by an
observation distribution $p[y(t)|h(t)]$.  In the voltage-sensing case,
for example, $y(t)$ will correspond to an observation of the
potentiometric fluorescence signal at some focal point on the
dendritic tree, which in turn provides some scaled, noisy, incomplete
information about $h(t)$, the full spatiotemporal voltage.

Methods for performing optimal inference and estimation in these
hidden Markov models are very well-developed in the statistics and
engineering literature \cite{RAB89,DFG01}.  In the classical setting,
where the dynamics $p[h(t)|h(t-1)]$ and observations $p[y(t)|h(t)]$
are linear and Gaussian, the solution to the filtering problem is
well-known: optimal inference may be implemented very efficiently via
the well-known recursive Kalman filter.  More generally, in the
non-Gaussian, nonlinear, continuous-space case, it is necessary to use
numerical techniques, of which the most generally robust and effective
are known as Sequential Monte Carlo methods, or ``particle filtering''
\cite{DFG01}.  Due to space constraints, we will not discuss the
details here; see, e.g., \cite{DFG01} for a general introduction, or
\cite{Ergun07,HP06,Vogelstein07,PAN07} for some previous applications
in neuroscience, or chapter 12 of our book in progress at
www.stat.columbia.edu/$\sim$liam/rkeb/book.pdf for a more in-depth
description.  Briefly, to perform optimal filtering, we want to
compute the conditional distributions $p[h(t)|Y]$ of our hidden
process at each time $t$, given the set $Y=\{y(1),\ldots,y(T)\}$ of
all of our available observations; just as in a standard
discrete-time, discrete-space hidden Markov model \cite{RAB89}, this
can be done using a recursive algorithm (a modified forward-backward
algorithm).  In the general case, this recursion must be computed
numerically; the particle filter implements this recursion using a
Monte Carlo technique (importance sampling).  Additionally, in each of
our applications, we may recover the parameters $\theta$ governing the
underlying system (e.g., intercompartmental conductances, membrane
resistances, observation noise variance, etc.) via an iterative
Expectation-Maximization (EM) algorithm, which is a general
statistical method for obtaining maximum likelihood parameter
estimates that is particularly well-suited to hidden Markov models
\cite{RAB89,DFG01}.  This EM approach may be implemented
\emph{without} any need for simultaneous direct intracellular
recordings; this greatly expands the scope of these methods.

While the hidden Markov framework is quite natural for each of our
applications, it should be emphasized that these particle filtering
methods do not work in an ``off the shelf'' manner.  Instead, we have
found in our preliminary efforts
\cite{HP06,Vogelstein07,PAN07,PanFerr08,Vogelstein08} that it is
essential to carefully consider the known biophysical details of the
underlying voltage and calcium dynamics, and of the imaging modality,
in order to obtain robust and computationally tractable results.  In
the following we will describe our preliminary results and provide
detailed development plans for each of our three major goals.



\subsubsection*{Aim 1: Extracting spike times from calcium; using
these inferred spike times to study population codes and infer
neuronal connectivity}

\noindent \textbf{Background}: A number of striking recent advances in
the development of calcium indicators, delivery techniques, and
microscopy technologies have facilitated imaging a wide array of
neural substrates \cite{SvobodaYasuda06}.  Calcium sensitive organic
dyes \cite{TsienRink82,YusteKatz92} have been targeted to populations
of neurons both in vivo and in vitro using bulk loading
\cite{BrusteinKonnerth03,StosiekKonnerth03} and electroporation
\cite{NagayamaChen07,NevianHelmchen07}.  Similarly, viral infection,
transgenics, and knock-ins have been used to genetically target
neurons with fluorescent proteins
\cite{MiyawakiTsien97,GriesbeckTsien01,NakaiImoto01,ShanerTsien04,GuerreroIsacoff05,OhkuraNakai05}.
In conjunction with the development of improved calcium indicators and
loading techniques, the development of 2-photon microscopy (2PM) now
enables the visualization of neurons deep within scattering tissue
\cite{DenkWebb90,OheimCharpak01,TheerDenk03,HelmchenDenk05,FlusbergSchnitzer05a}.

One of the most exciting applications of these novel calcium imaging
methods involves the direct observation of the collective dynamics of
populations of neurons
\cite{O'MalleyFetcho96,SmettersYuste99,CAR03,IkegayaYuste04,Kerr05,NiellSmith05,OhkiReid05,OhkiReid06,YaksiFriedrich07,NagayamaChen07,SatoSvoboda07,RootWang07,DombeckTank07,Holekamp08}.
The major roadblock in exploiting these large-scale imaging techniques
to their fullest potential is in the step of extracting the underlying
neuronal spike trains from the observed calcium fluorescence data;
this step is nontrivial because, as we discussed above, these signals
are noisy, subsampled, prone to saturation, and most importantly much
slower than the timescale of action potentials.  Previous efforts
towards inferring spike times from calcium fluorescence data have
involved linear deconvolution
\cite{YaksiFriedrich06,RamdyaEngert06,Holekamp08}, template matching
\cite{Kerr05}, or clustering methods \cite{SatoSvoboda07} and have
therefore been quite vulnerable to the nonlinear features of the data
(especially saturation of the fluorescence of the calcium indicator).
More importantly, none of these approaches attempt to incorporate our
considerable knowledge of the biophysics of calcium-sensitive
fluorescence \cite{NeherAugustine92}, especially with respect to
calcium flux following an action potential
\cite{RegehrTank94,TankDelaney95,RegehrAtluri95,FellerTank96,HelmchenSakmann96,MaravallSvoboda00,MajewskaYuste00,CornelisseMansvelder07,Gobel07b}.

Thus our first major goal is to develop optimal particle filtering
methods for inferring spike trains from these calcium-sensitive
fluorescence data.  We start by writing down our model for the
dynamics $p[h(t)|h(t-1)]$ and observations $p[y(t)|h(t)]$ in our
hidden Markov framework.  We will describe a simple illustrative
example of our approach.  The internal calcium concentration \Cat is
our hidden variable $h(t)$ here.  A first-order model for the
stochastic calcium dynamics is \eq d\Cat /dt = - \Cat / \tau_C + A n_t
+ \epsilon_t, \en where $\tau_C$ is a decay time constant; $A$ is a
magnitude constant that sets how much calcium current enters the cell
each time the cell spikes; $n_t$ represents the cell's (unknown) spike
train; and $\epsilon_t$ denotes an unobserved noise process.  (More
detailed models are of course possible; for further details and
extensions, see \cite{Vogelstein07}.)  We must also write down a model
$p(y_t | \Cat)$ for the observed fluoresence signal $y_t$ given
$\Cat$; this model can incorporate saturation of the fluorescence
signal, non-Gaussian shot noise, and so on.

\noindent \textbf{Preliminary results}: We have successfully developed
particle filtering methods for this system that are robust,
computationally efficient, and that consistently outperform simpler
linear methods in extracting the true underlying spike times and
calcium signals when tested on a wide variety of simulated examples
\cite{Vogelstein07}.  We have found that these algorithms can reliably
infer the expected timing of spikes at a temporal resolution greater
than that of the observed image frames, given single-trial data of
sufficiently high SNR; see Fig.~\ref{fig:array} for an example
application in which these simulation results can be extremely useful
in choosing the necessary imaging dwell time and sampling frequency
necessary to achieve a given level of accuracy in the recovered spike
train.  We are now testing these methods on real in vitro calcium
imaging data in which the true spike times (``ground truth'') have
been recorded simultaneously.  Our preliminary results are quite
promising (Fig.~\ref{fig:jv-real}): the particle smoother provides
accurate and robust estimates of the underlying spike times even in
highly-saturated, low-SNR regimes, again significantly outperforming
the comparable linear techniques.


\noindent \textbf{Project plan}: Our next step will be to
systematically quantify the robustness of these methods in as wide a
variety of real experimental preparations as possible.  We have begun
this process with in vitro data from the Yuste lab here at Columbia
(Fig.~\ref{fig:jv-real}); we next plan to examine in vivo recordings
from zebrafish (Engert lab, Harvard), where movement artifacts can be
minimized and available SNR is higher (due to the transparency of the
tissue), and finally we plan to analyze recordings from in vivo mouse
and rat preparations (in collaborations with groups at Janelia farms
and elsewhere).  We expect that a number of assumptions we have made
in our preliminary work will need to be relaxed in some experimental
settings; this will require modifications and extensions to our code,
as well as further optimization of the code's efficiency and
robustness.  In particular, we have focused so far on the problem of
temporally filtering the fluorescence signal observed in a given
spatial window at the soma, but we expect that gains in SNR should be
available via improved spatial or spatiotemporal filtering instead of
crude windowing.  We will in particular need to incorporate methods
for motion filtering (e.g., \cite{DombeckTank07}) in the in vivo
setting.

In parallel, we are developing faster methods.  The code used in
Figs.~\ref{fig:array} and \ref{fig:jv-real} is quite efficient; this
code (running on a non-optimized laptop computer) processes one second
of fluorescence in less than one second (i.e., real time).  However,
when many simultaneous recordings of calcium fluorescence signals from
multiple cells are available, faster methods are very useful.  We have
developed a method based on the classical Wiener deconvolution filter
(the optimal linear filter) that additionally enforces the sparseness
and non-negativity of the recovered spike train \cite{Vogelstein08};
the resulting filter output is a \emph{nonlinear} function of the
observed data $Y$.  As is well-known in the signal-processing
literature, incorporating non-negativity constraints can lead to
dramatic increases in the recovered SNR in deconvolution tasks
\cite{LLS04,HAP06,PanFerr08}.  The major advance here is that this
nonnegative deconvolution task can be solved quite efficiently; the
computation requires just $O(T)$ time, where $T$ is the number of
timesteps at which we want to estimate the firing rate, which is even
faster than the $O(T \log T)$ typically required for performing
(unconstrained) Wiener filtering in the frequency domain.  (Typically
nonnegative deconvolution requires $O(T^3)$ time, which would not be
useful here.)  This nonnegative deconvolution method does in fact
improve the SNR over the unconstrained Wiener filter
(Fig.~\ref{fig:jv-real}); while the particle filter provides
significantly greater SNR (particularly in the highly-saturated
fluorescence regime), the new method runs about two orders of
magnitude faster.  Therefore this nonnegative deconvolution method is
very well-suited to real-time preliminary analysis of large-scale
multineuronal calcium recordings, while the particle filter can be
used for more detailed, accurate analyses.

An important first biological application of our improved filtering
methods is to the semi-automated measurement of neuronal connectivity
via two-photon MNI-glutamate uncaging \cite{Vovan07}.  Previously,
invesigators have recorded postsynaptic potentials in a target neuron
following laser-uncaging stimulation of presynaptic cells, but there
has been a significant degree of uncertainty as to exactly when (or
whether) the presynaptic neurons fired following the stimulus.  By
imaging the presynaptic activity via calcium fluorescence measurements
and precisely quantifying the presynaptic spike times using our
improved filtering methods, we should be able to obtain greatly
improved estimates of the strength of the connectivity between any two
cells in the observed network.  (Similar applications to
channelrhodopsin-2- and halorhodopsin-based stimulation should also be
feasible in the near future \cite{Deisseroth05}.)  One key advantage
of our probabilistic filtering approach in this context is that we may
easily incorporate information about the stimulus timing to improve
the estimation of the resulting spike times \cite{Vogelstein07}.  We
have just begun this work in collaboration with the Yuste lab.

Finally, in order to take full advantage of large-scale observations
of populations of neurons via calcium imaging
\cite{CAR03,Kerr05,OhkiReid06,DombeckTank07,Holekamp08}, we plan to
combine our novel calcium-filtering methods with related statistical
population-coding methods we have previously developed and
successfully applied to model interacting multineuronal point-process
data \cite{PAN03d,SHO03,KP06,PILL07,PAN06c}.  By combining these two
statistical methodologies we will be able to directly quantify, for
the first time, the functional connectivity in these large neural
populations.  A first step is to quantify, through simulations similar
to those shown in Fig.~\ref{fig:array}, exactly how high a sampling
frequency and SNR are required to accurately infer the connectivity
between two cells.  Once the connectivity map has been estimated we
will calibrate and verify our results against the anatomical
connectivity measured using the methods described above.  Again, these
research directions will be pursued via experimental collaborations
both at Columbia and elsewhere.



\subsubsection*{Aim 2: Inferring voltage dynamics from noisy, intermittent
voltage-sensitive imaging data; using the estimated spatiotemporal
voltage to estimate biophysical parameters}

\noindent \textbf{Background}: Like the calcium-sensing methods
discussed above, voltage-sensitive technology has also progressed by
leaps and bounds over the past several years.  Voltage-sensitive dye
recordings have been pursued successfully at a multicellular scale
\cite{TKGA99,Petersen03,Benucci07,Seidemann08}, with single-unit
resolution \cite{Frost07}, and most recently with subcellular
resolution via two-photon techniques
\cite{Djurisic04,Canepari07,Fisher08,Djurisic08}.  More recently,
second-harmonic generation techniques have developed as another
promising voltage-sensing technology
\cite{Dombeck04,Sacconi06,NURIYA06}.  In addition, advances in
rapid-scanning technology (e.g., via acousto-optical deflection
\cite{Iyer06,Vucinic07}) promise increased spatiotemporal resolution.

The major roadblock currently confronting these exciting new
subcellular-resolution voltage-sensing techniques is low SNR.
State-of-the-art voltage-sensitive dye recordings at two-photon
spatial resolution and submillisecond temporal resolution (c.f.\ Fig.\
6 of \cite{Fisher08}, whose raw data we replot and analyze in
Fig.~\ref{fig:fisher} below) provide SNR that is at least an order of
magnitude less than that available with electrode recordings.  Of
course, imaging methods have two great advantages, namely their
ability to access small dendritic structures and the ability to
rapidly sample from many spatial locations.

Our second major aim is to exploit these advantages by developing
optimal spatiotemporal filtering methods for this data, with the
long-term goal of directly studying nonlinear dendritic computation.
As discussed above, casting this spatiotemporal filtering problem in
our hidden Markov framework is fairly straightforward.  In the
simplest case, we may assume that the dendritic tree is passive; thus
the hidden variable $h(t)$ comprises the vector of voltages $\vec
V(t)$ on the dendritic tree at time $t$, and the dynamics equation
$p[\vec V(t) | \vec V(t-1)]$ is supplied by the linear discretized
cable equation: 
\begin{equation}
  \vec V(t+dt) = \vec V(t) + dt A \vec V(t) + \vec \epsilon_t,
\label{eq:linear-dynamics}
\end{equation}
where again $\epsilon_t$ represents a noise source and
the sparse dynamics matrix $A$ here encodes the membrane leak at each
compartment as well as coupling currents between adjacent
compartments.  Note that we make the reasonable assumption here that
the anatomy of the imaged cell is known (or at least may be
reconstructed pot hoc); thus we know a priori which compartments are
adjacent, and specifying the matrix $A$ comes down to setting a few
resistivity coefficients, if the diameter of each compartment can be
estimated reliably.  (See \cite{HAP06} for further discussion.)  In
the case of strongly nonlinear dendritic dynamics, we have to augment
our hidden variable $h(t)$ with variables specifying the time-varying
open probabilities of active membrane currents, and the dynamics
matrix $A$ effectively becomes time-varying and voltage-dependent
itself; but the point to remember is that in either setting the
dynamics equation $p[h(t)|h(t-1)]$ follows directly from known
biophysical principles, once the necessary biophysical parameters have
been estimated.  Again, once we have specified a model for the
observations $p[y(t)|h(t)]$ we may proceed to construct the optimal
filter.

\noindent \textbf{Preliminary results}: We have obtained promising
results both in the quasilinear and strongly nonlinear regimes.  In
the linear regime discussed above, if the dynamics noise $\epsilon$
and observation noise can be approximated as Gaussian, then the
classical Kalman filter provides the optimal filter.  Due to the
sparse nature of the dynamics matrix $A$ here, the Kalman recursions
can be computed quite efficiently in this system.  A movie
illustrating the filter's performance given simulated noisy
raster-scanned voltage data on a dendritic branch is available at
\hp/mcknight.html; we see that despite the highly incomplete
(subsampled) observations, the filter recovers the underlying voltage
dynamics quite accurately.  We can also accurately recover the
parameters of the dynamics matrix $A$ via an Expectation-Maximization
approach; see Fig.\ 7 of \cite{HP06}.

In the case of spiking neurons, the linear dynamics model is
inappropriate, and we must augment our hidden variable $h(t)$ with
additional dynamical variables to obtain valid inferences.  For
example, the classical Hodgkin-Huxley model requires us to track four
independent variables (voltage plus three variables to control the
open fraction of the sodium and postassium gates), though some model
reductions can often be achieved \cite{EK86,DA01}.  We have begun with
single-compartmental models in this strongly nonlinear setting; a
successful application of a four-dimensional (modified Hodgkin-Huxley)
particle smoother to real fluorescence data is shown in
Fig.~\ref{fig:fisher}.  The results are quite encouraging.  In
simulated data, where ground truth is available, we have found that
these optimal filtering methods can increase spatiotemporal resolution
by an order of magnitude in low-SNR recordings \cite{HP06}; our
results on real data here are consistent with these gains so far.  The
next step is to calibrate our results against intracellularly recorded
voltage data, to ascertain how accurately we are recovering the
subthreshold voltage (not just the correctly-recovered superthreshold
voltages illustrated in Fig.~\ref{fig:fisher}).

Our major computational goal here will be to improve the efficiency of
the particle filtering method in the nonlinear, very-high-dimensional
spatiotemporal case.  As emphasized above, the particle filter is
quite effective when applied to models with a single or a few
compartments, and the Kalman filter is quite effective and extremely
computationally efficient in models with many compartments and
quasilinear dynamics.  To efficiently solve the nonlinear
spatiotemporal filtering problem in cases with very many compartments,
we will exploit the Kalman-based methods to improve the efficiency of
the more general particle methods; in particular, we can use the
output of the Kalman filter to automatically guide our sampling
strategy (more technically, we will choose our importance sampling
proposal density to be proportional to the Kalman-smoothed density in
the Monte Carlo update step of the forward sweep of the particle
filter).  Our preliminary results suggest that this strategy
significantly improves the filter's computational and statistical
efficiency; in fact, we found that a similar strategy was necessary
for the success of the particle filter in the calcium-sensitive
setting (see discussion on the ``conditional sampler'' in
\cite{Vogelstein07}).

Another important (and tractable) technical advance we are pursuing is
to incorporate the fact that some voltage indicators have their own
intrinsic dynamics, which must be incorporated into either the
dynamics equation (as an additional dynamical variable) or the
observation equation.  This modification will be particularly
important to take advantage of data from genetically-encoded voltage
indicators, which promise significant improvements in SNR and reduced
toxicity and bleaching \cite{Knopfel06}, but which currently operate
at a slower timescale than do voltage-sensitive dye or second-harmonic
generation signals.

We are currently pursuing two important biological applications of
these optimal filtering ideas.  First, as discussed above, we can use
an Expectation-Maximization algorithm to effectively infer the
biophysical parameters governing the dynamical (i.e., computational)
properties of the observed dendrites, on a compartment-by-compartment
basis.  We consider this automatic estimation of the cable equation
parameters (including both linear and nonlinear effects) a key step
for understanding dendritic computation in real neurons.  We have
successfully achieved this goal in a variety of simulated data; see
\cite{HAP06} for a number of very high-dimensional examples in the
idealized case of no observation noise (but possibly large dynamics
noise), \cite{HP06} for an analysis of how inference accuracy scales
with observation noise in a multicompartmental linear setting, and
Fig.~\ref{fig:hh-em} for an application to a simple nonlinear
Hodgkin-Huxley-type model given noisy observations.  We are currently
working to extend these methods to higher-dimensional (more channels
and more compartments) nonlinear data, before turning our attention to
real spatiotemporal voltage-sensing data \cite{Djurisic08}.  In
addition, we are working to extend our methods to automatically
estimate not just the density of the active membrane channels in a
given compartment, but also the kinetics of these channels; see
Fig.~\ref{fig:q-channels} for a successful preliminary result.

The second application is an extension of the automated connectivity
analysis discussed in the last section.  Once we have established that
two neurons are connected synaptically, we would like to quantify: 1)
where on the postsynaptic dendritic tree the synapses in question are
located, and 2) how strong each of these synapses are.  This is
schematically illustrated in the left panel of Fig.~\ref{fig:synapse};
we may model this data to first order by assuming that each
presynaptic spike is followed by an injection of current of unknown
magnitude and temporal shape into each compartment.  Thus we would
like to infer a set of spatiotemporal weights mapping the influence of
a presynaptic spike onto each postsynaptic \emph{compartment}, instead
of just the postsynaptic soma; this will provide a much more detailed
view of real synaptic computation on the dendritic tree.  Here the
linear Kalman model is appropriate, since the network will be in a
resting state during these in vitro experiments and therefore the
dendritic tree should be in a non-depolarized, quasi-linear state
(though shunting effects may play a role for large synchronous inputs
to nearby compartments; we have ignored shunting for now but plan to
incorporate these effects, as well as more sophisticated synaptic
depression and failure models, in the future).  Thus the synaptic
weights enter as simple regression terms and may be estimated easily
in the Kalman model.  In simulations (Fig.~\ref{fig:synapse}, right)
we have found that these simple Kalman inference methods are quite
efficient in extracting the true synaptic weights, even given heavily
subsampled and noisy data.  The performance of this method in
extracting the functional synaptic weights in living tissue can then
be calibrated directly against anatomical measurements of physical
synaptic size available via electron microscopy (or, in the future,
via recently-developed transgenic techniques \cite{Brainbow07}).  We
expect that this method, if successful, will lead to significant
insights into the detailed synaptic organization of local neural
circuits \cite{Song05}.


\comment{
multineuronal-scale: we expect that related filtering methods will be
very useful in these applications, and we hope to address these issues
in the future, though we do not plan to pursue this in the current
project.
}


\subsubsection*{Aim 3: Extracting voltage dynamics from
  calcium-sensitive imaging data; combining simultaneous
  voltage-sensing and calcium-sensing data.}

\noindent \textbf{Background}: Our final aim may be summed up more
briefly.  As discussed in the last section, a major prerequisite for
understanding dendritic computation is the observation (possibly
through indirect means) of the spatiotemporal voltage signal on the
dendritic tree.  Voltage-sensing techniques are currently hampered by
low SNR (although optimal filtering methods can at least partially
alleviate this problem, as we have demonstrated in
Fig.~\ref{fig:fisher} above).  On the other hand, spatiotemporal
calcium-sensing fluorescence signals may be currently recorded at
relatively high SNR \cite{Gobel07,Larkum08}, but calcium signals, by
their nature, provide only highly-thresholded information about the
underlying voltage, since calcium channels are inactive at
hyperpolarized voltages.  The obvious question is: can we exploit
these high-SNR superthreshold calcium observations to reconstruct (at
least partially) the subthreshold voltage signal?  More generally, can
we combine calcium and voltage measurements (where voltage
measurements may be available via imaging techniques, or whole-cell
patch recordings at the soma or apical dendrite, or even through dense
multielectrode recordings \cite{Petrusca07}) in an optimal way to
obtain good estimates of the subthreshold voltage?

\noindent \textbf{Preliminary results}: As usual, we begin by writing
down a model for the dynamics $p[h(t)|h(t-1)]$ and observations
$p[y(t)|h(t)]$.  For clarity, we stick with the simple linear dynamics
(Eq.~(\ref{eq:linear-dynamics})) for the spatiotemporal voltage.  Now
in general we need to write down a model for voltage-dependent calcium
influx, as well as for calcium extraction/buffering and for spatial
diffusion of calcium within the dendrite.  While we may reasonably
approximate the latter terms with linear calcium dynamics, the
voltage-dependent influx term will necessarily be nonlinear, and so
particle filtering methods (treating the joint vector of the spatial
calcium concentrations and voltages as the hidden variablke $h(t)$)
are required.  

However, a shortcut is available which we have found to be quite
effective.  We use a first-order model for the calcium dynamics: \eq
dC_x(t)/dt = -C_x(t) / \tau_C + k \left[C_{x+dx}(t) -2C_x(t)
+C_{x-dx}(t) \right] + f[V_x(t)] + \epsilon_{xt}, \en where $C_x(t)$
denotes the calcium concentration in compartment $x$ at time $t$, $k$
is a (dimensional) diffusion constant, $k \left[C_{x+dx}(t) -2C_x(t)
+C_{x-dx}(t) \right]$ represents diffusion within the dendrite (this
assumes that $i$ corresponds to an interior compartment of a linear
segment of the dendrite, and may be easily modified in the case of a
boundary or branching compartment), and $f(V)$ is a nonlinear function
corresponding to voltage-dependent calcium influx.  (Note that we have
approximated the calcium channel dynamics as instantaneous here; this
is another assumption that can be relaxed.)

Now the important point is that the linear $k$ and $1\tau_C$ terms
here are relatively small; as we have emphasized above, in the
subthreshold regime the calcium concentration changes much more slowly
than the voltage.  We can take advantage of this fact: if we sample
sufficiently rapidly along the dendritic tree, then we may obtain
$C_x(t)$ (up to some observation noise) and then numerically subtract
the estimated $dC_x(t)/dt$ from the linear terms on the right hand
side of our dynamics equation to obtain, finally, our observation
$y(t)$, which will correspond to a nonlinear measurement $f[V_x(t)]$
of the voltage.  Of course, this will be a noisy measurement; the
variance of this noise can be computed given the variance of the
dynamics noise $\epsilon$ and the calcium-sensitive observation
fluorescence noise.  Performing optimal inference given quasilinear
voltage dynamics and nonlinear, noisy voltage observations can be done
very efficiently via modified Kalman methods we have developed
\cite{PanBook,Vogelstein08}; Fig.~\ref{fig:ca_to_v} illustrates a
successful application of these methods to simulated data.  (See also
\cite{HP06} for an example application to a single-compartment model
with more detailed (nonlinear) calcium channel and voltage dynamics.)
We see that the goal of extracting voltage transients from noisy
calcium measurements is feasible, although as expected details of the
subthreshold voltage are lost, since as discussed above calcium
measurements simply do not provide information about the subthreshold
voltage.

\noindent \textbf{Project plan}: As in the previous aim, we have
established that Kalman-based methods are effective for
high-dimensional (multicompartmental), quasilinear dynamics, while
particle filter methods are effective for lower-dimensional cases.
The next step is to combine the advantages of the two approaches, by
utilizing the Kalman-based methods to improve the Monte Carlo
importance sampling efficiency of the particle filter, while again
relaxing our assumptions about the dynamics as much as computationally
possible.  In addition, incorporating voltage observations (from
voltage-sensitive imaging, or electrode recordings, or both) in this
setting is quite straightforward, and will significantly improve the
accuracy of our subthreshold voltage estimates.  Once the methods have
been carefully tested on numerical simulations, we will again (with
our experimental collaborators) calibrate the results against ``ground
truth'' electrophysiological data recorded via dendritic patch clamp.



\subsubsection*{Conclusion and significance}

\cite{HAP06,Vogelstein07,HP06,PAN07}.  (Preprints available at
www.stat.columbia.edu/$\sim$liam/mcknight.html.)


These analysis techniques have the
potential to dramatically extend the power of currently available
imaging methods.  For example, the linear deconvolution methods that
are currently used to quantify calcium signals in spiking neurons
require experimentalists to limit their attention to the nonsaturating
(linear) range of their calcium indicators.  By handling saturation
effects properly (c.f.\ Fig.~\ref{fig:jv-real}), our optimal filtering
techniques can greatly expand the dynamical range of current recording
methodologies.  Similar dramatic gains in SNR can be obtained by
properly filtering voltage-indicator signals (Fig.~\ref{fig:fisher}).
These technological improvements will open up a wide new range of
experimental possibilities, allowing us to more fully exploit these
exciting new imaging techniques.

Finally, these algorithms have modest
computational demands; the algorithms work on standard desktop
computers and can therefore easily be employed in real time during
imaging experiments, to aid online experimental design.

we expect progress on the first aim to be fastest, due to the relative
popularity of the calcium-sensing technique and because our goals only
require temporal filtering, not full spatiotemporal filtering.


\noindent \textbf{Access to the new technology}: A major goal of this
project is to provide free, open-source software tools that can be
used by labs throughout the world.  Our code will be made public and
integrated where possible with currently-employed fluorescence
image-processing software.  Preliminary versions of some codes are
already available online or on request
\cite{HAP06,HP06,Vogelstein07,Vogelstein08}.  This code will be
well-documented internally and the underlying statistical methodology
and implementation details will be described in journal publications
as well as a planned review article and in chapter 12 of the textbook
in preparation \cite{PanBook}.


\clearpage
\pagestyle{empty}

\noindent \textbf{Preliminary data}

\begin{figure}[b!]
\begin{center}
\epsfxsize=6in \epsffile{array.eps}
\caption{\small Dependence of spike train inference on noise and
  intermittency of simulated calcium fluorescence observations.  All
  panels show the fluorescence observations (top plots; black dots);
  the true and inferred \Cat (middle plots; gray and green $\pm$
  posterior standard deviation, respectively); and the true and
  inferred spike times (bottom plots; gray and green, respectively).
  From bottom to top, each row has increasing variance $\sigma^2_0$ on
  the observation noise.  From right to left, each column has
  observation sampling frequency decreasing by the same factors.
  Diagonal arrows indicate equal total SNR per unit time.  Note that
  inference remains accurate over a wide range of SNR; furthermore, by
  the optimality of the particle filter, these results provide an
  upper bound on inference accuracy at the SNR/intermittency levels
  shown here \cite{Vogelstein07}.}
  \label{fig:array}
\end{center}
\end{figure}

\begin{figure}[p]
\begin{center}
\hbox{
\hspace{-1cm}
\epsfxsize=7in
\epsffile{Fig_jv_real_ca.eps}
}
\caption{\small Accurate inference of spike times given noisy,
  saturating real calcium fluorescence signals.  Top panel: observed
  fluorescence and true spike trains.  (Data recorded in vitro from a
  pyramidal cell in mouse somatosensory cortex; voltage recorded via
  whole-cell patch; calcium imaged via epifluorescence sampled at 60
  Hz with fura-2; spike times defined by voltage threshold crossing.
  Data courtesy of B.\ Watson, A.\ Packer, and R.\ Yuste.)  Note the
  strong saturation in the fluorescence signal visible around 5-6 sec.
  Middle panels: comparison of the spike times inferred using three
  methods: Wiener filtering (the optimal linear filter); the fast
  sparsening nonnegative deconvolution filter discussed in the text;
  and the optimal particle filter.  Note that imposing non-negativity
  constraints significantly improves the SNR and temporal precision of
  the recovered spike times (at minimal computational cost), and the
  optimal particle filter improves the SNR significantly further.
  Light green bars indicate the 5th and 95th percentile of the
  inferred spikes; thus we see that even when the fluorescence signal
  is less informative due to saturation (at 5-6 sec; c.f.\ reduced
  accuracy of the Wiener filter in this range), the particle filter
  still returns reasonable errorbars.  Gray line-dots indicate the
  true spike times, to facilitate comparison; red trace indicates
  times when Wiener filtering returns a negative firing rate.  Bottom:
  calcium signal inferred by the particle filter.}
\label{fig:jv-real}
\end{center}
\end{figure}

\begin{figure}[p]
\begin{center}
\epsfxsize=6in
\epsffile{Fig_fisher_data.eps}
\caption{\small Optimal voltage smoothing given noisy real
single-trial voltage-sensing fluorescence recordings.  Top: observed
data (raw potentiometric two-photon fluorescence data from Fig.\ 6 of
\cite{Fisher08}; data courtesy of J.\ Fisher and B.\ Salzberg).  The
neuron was stimulated electrically at 16 Hz, as indicated by the black
square pulses at the bottom of the bottom panel.  Fluorescence data
have been artificially redrawn on a voltage scale to allow direct
comparison with the recovered voltage.  Note the relatively low SNR.
Middle: voltage estimated by optimal linear Kalman filter ($\pm 1$
posterior standard deviation); since the underlying voltage dynamics
are strongly nonlinear (spiking) here, the best linear filter is
suboptimal, though this method does significantly improve the
effective SNR of the recording.  Bottom: voltage estimated by optimal
particle filter.  The dynamics equation $p[h(t)|h(t-1)]$ was given by
a modified Hodgkin-Huxley model; the observation was taken to be
linear with Gaussian noise (the linear-Gaussian assumption has been
checked and found to be reasonable on non-spiking data; results not
shown).  All free parameters (including nonlinear channel densities,
subthreshold noise variance, etc.) were fit directly to the data via
Expectation-Maximization.  Note the sharply improved precision of the
particle filtered signal.  Gray dots indicate raw data, as in top
panel, for comparison.}
\label{fig:fisher}
\end{center}
\end{figure}


\begin{figure}[t!]
\begin{center}
\epsfxsize=6in
\epsffile{q-hh-em.eps}
\caption{\small Inferring biophysical parameters from noisy simulated
  data \cite{HP06}.  A: Noisy observed voltage data from a simulated
  single-compartment Hodgkin-Huxley-type neuron.  B: True underlying
  voltage (black dashed line) resulting from current pulse injection
  shown in E. The gray trace shows the mean inferred voltage after
  inferring the parameter values in C. C: Initial (blue +) and
  inferred parameter values (red $\times$) in percent relative to true
  values (gray bars; $\bar g_{Na} = 120$ mS/cm$^2$, $\bar g_{K} = 20$
  mS/cm$^2$, $\bar g_{Leak} = 3$ mS/cm$^2$, $R_m = 1$ mS/cm$^2$).  At
  the initial values the cell was non-spiking. D: Magnified view
  showing data, inferred and true voltage traces for the first
  spike. Thus, despite the very high noise levels and an initially
  inaccurate, non-spiking model of the cell, knowledge of the channel
  kinetics allows accurate inference of the channel densities and very
  precise reconstruction of the underlying voltage trace.}
\label{fig:hh-em}
\end{center}
\end{figure}


\begin{figure}[t!]
\begin{center}
\epsfxsize=5in
\epsffile{Fig_q_channels.eps}
\caption{\small Inferring channel properties from simulated voltage
  traces.  Top: an increasing series of step pulses were injected into
  a simulated single-compartment neuron with both active
  (voltage-sensitive) and passive channels.  Black trace indicates the
  true voltage; red indicates the model prediction on this training
  data.  Middle: nonparametric inference of the voltage-dependent
  properties of a potassium channel.  In this case the open fraction
  $o(t)$ of the potassium channel obeyed the voltage-dependent
  Hodgkin-Huxley-like dynamics $o(t)=n(t)^4$, with $dn/dt = -a(V)n(t)
  + b(V)$.  The true voltage-dependence functions $a(V)$ and $b(V)$
  are shown in black; our estimates of these functions, given just the
  data observed in the top panel (red trace, $\pm$ posterior standard
  deviations in blue), were computed using an efficient nonparametric
  regression barrier method that enforced the nonnegativity constraint
  $a(V) \geq 0$ \cite{PanBook}.  The kinetics functions $a(V)$ and
  $b(V)$ are recovered with high accuracy here.  Bottom: comparison of
  predicted and true voltages given a novel input current stimulus;
  note the close match between the black (true) and red (predicted)
  traces.  Predictions remain robust and accurate even when the model
  is estimated assuming the wrong nonlinearity (e.g.,
  $o(t)=n(t)^\alpha, \alpha \neq 4$); data not shown.}
\label{fig:q-channels}
\end{center}
\end{figure}


\begin{figure}[t!]
\begin{center}
\hbox{
\hspace{-1.8cm}
\epsfxsize=2in
\epsffile{synaptic-weights.eps}
\epsfxsize=5.6in
\epsffile{Fig_synapse_detection.eps}
}
\caption{\small Detecting the location and strength of synapses on a
  dendritic branch.  Left: schematic of proposed method.  By observing
  a (possibly noisy, subsampled) spatiotemporal voltage signal on the
  dendritic tree, we can infer the strength of a given presynaptic
  cell's inputs at each location on the postsynaptic cell's dendritic
  tree.  Right: illustration of the method applied to simulated data
  \cite{PanFerr08}.  Top: a simulated neuron received two presynaptic
  inputs with known spike times; one input was excitatory (indicated
  in red), and one was inhibitory (blue).  Second panel: these inputs
  led to a noisy spatiotemporal voltage response on an observed linear
  segment of the postsynaptic neuron's dendritic tree; colormap
  indicates true spatiotemporal voltage in mV.  This true voltage was
  not observed directly; instead, we only observed a noisy,
  spatially-rastered subsampled version of this signal (third panel;
  raster scanning is meant to emulate linescans in a two-photon
  setting).  Note the very low effective SNR here.  Bottom: by
  applying an efficient Kalman-based estimation method \cite{PanBook},
  we were able to recover the true synaptic weights on the dendrite,
  given only the noisy, subsampled 500 ms of voltage data shown in the
  third panel.  Solid traces indicate the true spatial weights; dashed
  errorbars show inferred weights, $\pm$ the posterior standard
  deviation (this posterior uncertainty is significant, given the
  noisy nature of the observed data).  Similar results are seen when
  inferring spatiotemporal synaptic weights (instead of purely spatial
  weights with known temporal effects, as shown here; data omitted for
  clarity).}
\label{fig:synapse}
\end{center}
\end{figure}


\begin{figure}[p]
\begin{center}
\epsfxsize=5.4in
\epsffile{Fig_ca_to_v.eps}
\caption{\small Inferring spatiotemporal voltage from noisy,
subsampled simulated calcium-sensing recordings.  Top: true
spatiotemporal voltage on a simulated dendritic branch.  Compartment 1
of this model neuron recieved a periodic current input; color
indicates voltage response.  Panel 2: observed data.  We used the
subtraction method described in the text to reduce the noisy calcium
measurement to a noisy, nonlinear measurement of the voltage, $y(t) =
f[V(t)] + \eta_t$, where $\eta_t$ in this case is independent Gaussian
noise.  Here the voltage-dependent calcium current $f(V)$ had an
activation potential at -20 mV (i.e., the calcium current is
effectively zero at voltages significantly below $-20$ mV; at voltages
$>10$ mV the current is ohmic).  We observe a spatially-rastered
subsampled version of this $y(t)$ signal here.  Note, as in
Fig.~\ref{fig:synapse}, the low effective SNR.  Panel 3: inferred
voltage signal given noisy, subsampled observations $y(t)$ shown in
panel 2.  Bottom panel: posterior standard deviation of the voltage
estimate shown in panel 3.  Note that only a few $y(t)$ observations
--- those corresponding to superthreshold calcium currents $f[V(t)]$
--- provide most of the information here; when no superthreshold
observations are available (e.g., between times 0.04 and 0.06 sec),
the estimated voltage misses the small fluctuations in the true
voltage response, although the superthreshold spatiotemporal voltage
transients are reconstructed quite accurately.  Despite the high
dimensionality ($>10^4$) of the inferred spatiotemporal voltage here,
the optimal filter requires just a few seconds to run on a laptop
computer, and is therefore feasible for use in real-time imaging
experiments.}
\label{fig:ca_to_v}
\end{center}
\end{figure}

\clearpage
%\bibliographystyle{apalike}
\bibliographystyle{abbrv}
\bibliography{../../mybib,./jv-paper}


\clearpage

\noindent \textbf{Proposed Budget}

The proposed research does not require any specialized equipment
beyond standard desktop computers.  Instead, we need talented and
devoted personnel to help us carry out the research successfully.
This award will partially support two personnel: a postdoctoral
researcher and a graduate student.

Postdoc salary: \$45K + 3\% increase/year

Ph.D.\ student stipend: \$28K + 3\% increase/year

Computer purchases for postdoc and student in the first year: 2 x
\$3K.

(Support for fringe benefits for the postdoc and tuition for the
Ph.D.\ student are available through other funding sources.)


\clearpage

\noindent \textbf{Other sources of funding}

\vs

NSF Faculty Early Career Development (CAREER) Award.

Title: CAREER: Using advanced statistical techniques to decipher the
    neural code

PI: Paninski

Dates: 6/2007-5/2012

Amount: \$100,000/year

\vs


NEI R01 EY018003

Title: Collaborative Research in Computational Neuroscience: Complete
 functional characterization of a population of retinal ganglion cells

PIs:  Paninski and Simoncelli, E. and Chichilnisky, E.

Dates: 9/2006-8/2011

Amount: \$130,000/year to Paninski; \$475,000/year total

\vs


Alfred P. Sloan Research Fellowship 

PI: Paninski

Dates: 9/2007-8/2009

Amount: \$46,000

\vs


Gatsby Initiative in Brain Circuitry (Columbia University institutional award)

Title: Ethological relevance in neural coding of communication sounds

PIs: Paninski and Woolley, S.

Dates: 9/2006-8/2008

Amount: \$25,000 to Paninski; \$50,000 total


\vs

Submitted: McKnight Scholar Award

Title: Using advanced statistical techniques to decipher population
codes

PI: Paninski

Dates: 2008-2011

Amount: \$225,000


\vs

Submitted: Klingenstein Fellowship Awards in the Neurosciences

Title: Understanding neural population codes under normal
  and epileptic conditions

PI: Paninski

Dates: 2008-2011

Amount: \$150,000


\vs


Institutional support from Columbia University

Startup: \$10,000

Salary: \$85,000 per 9 months

Research funds: \$7,000/year


\vfil \vfil


\end{document}
