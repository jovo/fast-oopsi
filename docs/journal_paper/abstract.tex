Experiments often yield measurements of variables that are naturally constrained to be nonnegative. In such scenarios, it may be desirable to filter (or deconvolve) the observations to find the most likely trajectory of the nonnegative variable, given the noisy observations. Here, we develop a computationally-efficient optimal filter for a certain subset of nonnegatively constrained deconvolutions.  Specifically, for any nonnegative variable that is filtered by a matrix linear differential equation and observed with independent, log-concave noise, we can infer the optimal nonnegative trajectory via straightforward interior-point methods in $O(T)$ (linear) time, as opposed to more standard approaches requiring $O(T^3)$ time (where $T$ is the total number of time steps).  The key is to make use of the tridiagonal structure of the Hessian of the log-posterior here, which allows us to perform each Newton iteration in linear time.  We apply this filter to an important problem in neuroscience: inferring a spike train from noisy calcium fluorescence observations. We demonstrate the filter's improved performance on simulated and real data. In conclusion, we propose that this filter is readily applicable for a number of real-time applications, including spike inference from simultaneously-observed large neural populations.

