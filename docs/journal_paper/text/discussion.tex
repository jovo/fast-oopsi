We show here that for certain nonnegative deconvolution problems, we can derive an algorithm that is both optimal and efficient.  More specifically, our algorithm may be applied to any model with a nonnegative signal that is linearly filtered by a matrix linear ordinary differential equation.  We apply this approach to the problem of inferring the most likely spike train given noisy calcium sensitive fluorescence observations (c.f. Fig.\ \ref{fig:demo}), and demonstrate, in simulations, that the optimal nonnegative filter outperforms the optimal linear (i.e., Wiener) filter in both slow and fast firing rate regimes (c.f. Fig.\ \ref{fig:comp}).  Furthermore, when applied to data from a live cell, the optimal nonnegative filter outperforms a fast projection pursuit regression filter, which constrains the inferred spike train to be nonnegative integers (c.f. Fig.\ \ref{fig:real}). On the other hand, the nonnegative filter is based on a linear observation model, and therefore suffers a loss of precision in the presence of strong saturation effects, in contrast to the optimal nonlinear particle filter (c.f. Fig.\ \ref{fig:real}).    

The implications of these results are severalfold.  First, it seems as if there is no reason to use the Wiener filter for scenarios in which our algorithm may apply.  Second, as our filter is so efficient, it may be used for many real-time processing applications.  Specifically, upon simultaneously imaging a population of neurons \cite{IkegayaYuste04, NiellSmith05, OhkiReid05, YaksiFriedrich06, SatoSvoboda07}, our filter may be applied essentially online.  This could greatly expedite the tuning of important experimental parameters --- such as laser intensity --- to optimize signal-to-noise ratio for inferring spikes.  Third, the parameters estimated from this filter may be used to initialize the parameters of the optimal nonlinear particle filter, which may then be used offline, to further refine the spike train inference. % Because the optimal nonlinear particle filter performs in approximately real-time (making it $\sim 100$ fold slower than the filters developed here), it may be run overnight on all the neural data collected in a daily experimental session. 
%Future work will consider multidimensional models for this application, incorporating both more sophisticated calcium models, and spatial filtering for extracting the fluorescence signal, obviating the need for additional algorithms for image segmentation. 

