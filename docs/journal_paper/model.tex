We assume a simple discrete-time state-space model relating spikes, $n_t$, baseline subtracted intracellular calcium concentration, denoted by $C_t$, and fluorescence measurements, $F_t$.  First, we assume a linear relationship between $F_t$ and $C_t$, with gaussian noise.  Second, we assume that $C_t$ jumps after each spike, and then decays according to some time constant.  Finally, we assume that spikes are distributed according to a Poisson process. The above assumptions lead to the follwoing model:

%. (1) spikes follow Poisson statistics with rate $\lam \Del$, (2) $C_t$ decays exponentially with decay $\gamma$ to baseline $\nu$, but jumps by $\rho$ after each spike, and (3) fluorescent observations are linear functions of $C_t$ with additive Gaussian noise with variance $\sig^2$.  Together, these assumptions imply the following model:

\begin{align}
F_t &= \alpha C_t + \beta +  \varepsilon_t, \qquad &\varepsilon_t \sim \mathcal{N}(0,\sig) \label{eq:obs} \\
C_t  &= \gamma  C_{t-1} + n_t,  \qquad &n_t \sim \text{Poisson}(n_t; \lam \Del) \label{eq:trans}  
\end{align}

\noindent where $\alpha$ and $\beta$ set the fluorescence baseline and offset respectively, $\sig$ indicates the variance of the noise, and $\gamma$ sets the decay rate. %Note that   . To enforce identifiability, we will let $\alpha=1$ and $\beta=0$, without loss of generality, as explained below.


