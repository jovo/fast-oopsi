%Experiments often yield measurements of variables known to be nonnegative, due to physical constraints. Examples arise in a wide variety of fields, including both audio and image signal processing. Therefore, determining the most likely trajectory of the nonnegative variable, given the observations,  requires a filter that performs a nonnegative deconvolution. Furthermore, by imposing such a nonnegativity constraint on the solution, these constraints also regularize the resulting inference \cite{LeeSeung99, LeeSeung01, HuysPaninski06}, by combating against ``ringing'' overfitting effects. Nonnegative deconvolution is closely related to another set of problems, called nonnegative matrix factorization, which decomposes a nonnegative matrix into the product of two nonnegative matrices \cite{OGradyPearlmutter06}.
%
%Unfortunately, imposing nonnegative constraints on deconvolution or matrix factorization remains difficult.  Although a number of methods have recently been introduced \cite{PortugalVicente94, LeeSeung99, LeeSeung01, LeeNg06}, they all scale polynomially with the number of time steps.  Here, we show that in some important special cases, we can solve these problems in linear time.  The major restriction we place on the filter is that it is the solution of a matrix linear ordinary differential equation. 
%
%One important example comes from neuroscience.  Calcium-sensitive fluorescent indicators are becoming an increasingly popular tool for visualizing neural spiking activity, both in vivo and in vitro \cite{YasudaSvoboda04}. Spikes cause a sharp rise in intracellular calcium concentration, $\Ca$, which is reported by a concurrent rise in fluorescence activity \cite{YasudaSvoboda04}.  As spikes are nonnegative entities, and $\Ca$ and spikes are related by a linear matrix differential equation, this data is indeed a special case of the more general class of problems described above.  We therefore develop an optimal and efficient algorithm for inferring spikes from noisy fluorescence observations, which we cast in the language of a nonnegative deconvolution problem. 
%
%The remainder of this paper is organized as follows. In Section \ref{sec:methods} we describe a highly efficient and optimal nonnegative filter for solving the kinds of problems described above, by making use of two tools. First, we use an interior-point technique for dealing with the nonnegativity constraint, in which we iteratively solve a series of related log-concave problems. Second, when the likelihood to be maximized adheres to our constraints, we can use an efficient algorithm to solve each iteration in $O(T)$ time, where $T$ is the total number of time steps. In Section \ref{sec:results}, we first use simulations to compare this filter with a few other possible filters: the optimal linear (i.e., Wiener) filter, and a fast version of an algorithm referred to as projection pursuit regression (PPR), adapted to our model of interest.  Then, we apply these fast filters to an example fluorescence time-series recorded from a live neuron \emph{in vitro}, and compare with an optimal nonlinear particle filter \cite{BJ08}. Finally, in Section \ref{sec:dis}, we discuss some applications and extensions.
%
%
%As mentioned above, using calcium-sensitive fluorescent indicators is becoming increasingly popular, as it enables simultaneously observing the activity of many (e.g., up to $500$) neurons.  Unfortunately, the image quality is typically relatively poor. To maximize the utility of this data, it would therefore be desirable to have an optimal filter, that would provide a spike train given only fluorescence measurements.  Recent work towards inferring spike trains from such data have made significant advances \cite{SmettersYuste99, KerrHelmchen05, YaksiFriedrich06, HolekampHoly08}, but none have utilized the fact that spikes are nonnegative. We therefore develop such a filter, in the larger context of nonnegative filter theory.

%\paragraph{Constructing the optimal nonnegative filter} \label{sec:nng}

